\documentclass{article}
\usepackage[utf8]{inputenc}
\usepackage{setspace}
\usepackage{natbib}
\usepackage{graphicx}
\usepackage{url}
\onehalfspacing

\title{Proof of Concept: User Stories}
\author{Matthew Clark\\43695841}
\date{\vspace{-5ex}}

\begin{document}
\maketitle
\newpage
\section*{User Story}
As a historical musicologist \& computer scientist, I want to be able to be able to develop a data set which contains the metadata of popular Techno songs so I can both quantitatively choose significant songs as analytical tools for further research and for the development of a data set to be used for machine learning.
\section*{Acceptance Criteria}
I need to be able to:
\begin{enumerate}
    \item Be able to successfully obtain data using the Discogs' API
    \item Develop a scope of techno songs that I can feasibly obtain using the API
    \item Automate a looping process that goes through the API and pulls the required metadata and stores the metadata locally
    \item Be able to convert the data into a re-usable format (Eg: JSON, CSV)
    \item Clean up the data for usability and re-usability.
\end{enumerate}
The data I want to be able to obtain includes:
\begin{itemize}
    \item Artist Name
    \item Track Title
    \item Release Title
    \item Track number
    \item Year Released
    \item Release Label
    \item Genre(s)
    \item Popularity Ranking
\end{itemize}
\section*{Project Management}
My project management system will be Trello, and can be found using the following link:\\
\url{https://trello.com/b/Vkgni6QN/foar705-musicologist}
\section*{Quality Assurance}
To maintain quality assurance, I will be using my learning journal to document any errors and bugs that I encounter, and use it to move my project forward based on my user story.
\section*{Scope}
My user story maintains my 2nd scope proposed in my previous PoC submission.
\end{document}
